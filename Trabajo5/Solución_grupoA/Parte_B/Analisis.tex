\documentclass[6pt]{report}
\usepackage[utf8]{inputenc}
\usepackage{anysize} 
\usepackage[spanish]{babel}
\usepackage{longtable}
\usepackage{lscape}
\usepackage{multirow} % para las tablas
\usepackage{graphicx}
\usepackage{anysize}
\usepackage{ragged2e}
\usepackage{hyperref}
\marginsize{1.50cm}{1.50cm}{1.50cm}{1.50cm} 
\usepackage{Sweave}
\begin{document}
\centering
\newcommand{\titulo}{ Universidad Nacional de Loja \\ \ \\ Ingeniería en Sistemas \\ \ \\}
\newcommand{\fecha}{\today}



\Sconcordance{concordance:Analisis.tex:Analisis.Rnw:%
1 12 1 1 0 48 1 1 2 41 0 1 2 1 1 1 2 7 0 1 2 1 1}

\pagestyle{empty}


\begin{center}\includegraphics[height=3cm, width=4cm]{logo-unl.jpg}\end{center}


\Huge\bf\titulo
\raggedright\textbf{Autores:}
\begin{itemize}
\item \href{http://www.iralis.org/?q=node%2F10&paso=10&letra=&id=7221}{ECINF7221}
\item \href{http://www.iralis.org/?q=node%2F10&paso=10&letra=&id=7220}{AFINF7220}
\item \href{http://www.iralis.org/?q=node%2F10&paso=10&letra=B&id=7326}{AFINF7326}
\item \href{http://www.iralis.org/?q=node%2F10&paso=10&letra=E&id=7219}{ECINF7219}
\item \href{http://www.iralis.org/?q=node%2F10&paso=10&letra=G&id=7206}{AFINF7206}
\end{itemize}
\raggedright\textbf{Docente:\\}
\begin{itemize}
\item \href{http://www.iralis.org/?q=node%2F10&paso=10&letra=O&id=4796}{ECINF4796}
\end{itemize} 

\flushleft
\Large\rm\fecha

\newpage



\textbf{Es aplicable la ingeniería de software cuando se elaboran webapps? Si es así, cómo puede modificarse para que asimile las características únicas de éstas?}

Si es aplicable, a pesar de ser webapps tienen un desarrollo similar, pero el enfoque de desarrollo debe centrarse en los requisitos propios de las webapps como:\\
- Concurrencia \\ 
- Rendimiento \\
- Disponibilidad \\
- Uso intensivo de redes \\
- Carga impredecible \\
- Orientadas a los datos \\

\textbf{Un breve descripción del datast Titanic}\\
Este conjunto de datos proporciona información sobre el destino de los pasajeros en el primer viaje fatal del trasatlántico Titanic, que se resumen de acuerdo a la situación económica, el sexo, la edad y la supervivencia.\\

\textbf{Mostrar el dataset}
\begin{Schunk}
\begin{Sinput}
> Titanic
\end{Sinput}
\begin{Soutput}
, , Age = Child, Survived = No

      Sex
Class  Male Female
  1st     0      0
  2nd     0      0
  3rd    35     17
  Crew    0      0

, , Age = Adult, Survived = No

      Sex
Class  Male Female
  1st   118      4
  2nd   154     13
  3rd   387     89
  Crew  670      3

, , Age = Child, Survived = Yes

      Sex
Class  Male Female
  1st     5      1
  2nd    11     13
  3rd    13     14
  Crew    0      0

, , Age = Adult, Survived = Yes

      Sex
Class  Male Female
  1st    57    140
  2nd    14     80
  3rd    75     76
  Crew  192     20
\end{Soutput}
\end{Schunk}

\textbf{¿Cuál es el número total de casos en el dataset?}
\begin{Schunk}
\begin{Sinput}
> sum(Titanic)
\end{Sinput}
\begin{Soutput}
[1] 2201
\end{Soutput}
\end{Schunk}
El número total de casos en el dataset Titanic es: 2201
\end{document}
